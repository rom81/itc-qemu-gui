\documentclass{article}
\usepackage[utf8]{inputenc}
\usepackage{graphicx}
\usepackage{vhistory}
\usepackage[margin=1in]{geometry}
\usepackage{indentfirst}
\usepackage{hyperref}
\hypersetup{
    colorlinks=true,
    linkcolor=blue,
    filecolor=magenta,      
    urlcolor=cyan,
}
% gives version of GUI to make further editions of this document easier to make
\newcommand{\version}{v.0.2}
\newcommand{\code}[1]{\texttt{#1}}

\title{ITC-QEMU-GUI \version{} Documentation}
\author{}
\date{\today}

\begin{document}

\maketitle
\tableofcontents



% Start of the revision history table
\begin{versionhistory}
  \vhEntry{v.0.1}{July 8, 2020}{Ian Peitzsch|Michael Hoefler}{Created}
  \vhEntry{v.0.2}{August 3, 2021}{Rachel Misbin}{Revised}
\end{versionhistory}

\newpage

\section{Introduction}

\subsection{Purpose}
The purpose of this document is to explain the installation process as well as the use and functionality of ITC QEMU GUI \version{}. This codebase can be publicly accessed at \url{https://github.com/rom81/itc-qemu-gui/tree/master}.

\subsection{Installation}
Before installing the GUI, make sure you have the following dependencies installed on your machine. These are all available to install from most Linux package managers.
\begin{itemize}
    \item \code{git}
    \item \code{build-essential}
    \item \code{python3}
    \item \code{python3-virtualenv}
\end{itemize}

To install the GUI, you should first create a custom build of QEMU using the provided \code{qemu\_install.sh} script. This is necessary because the GUI takes advantage of some commands which are custom and are not available in vanilla QEMU. After using the script to build QEMU, the \code{startup.sh} script is used to set up and run the GUI. Since the script must be run in the context of the current terminal, make sure to run it using \code{source startup.sh}. The script will install a Python virtual environment, install necessary packages in the environment, and run the GUI. This command may be used each time you start the GUI.
To start using the GUI, you should first run an instance of QEMU, making sure QMP is accessible. A simple example is listed below (which assumes that you have already created a bootable alpine image). If you are unsure how to get started with QEMU, refer to IV\&V's QEMU manual, itc-qemu-manual. \newline \newline
\code{\$ qemu-system-x86\_64 -qmp tcp:127.0.0.1:55555,server,nowait  ../alpine/alpine.img}
\newline

\begin{center}
    \includegraphics[]{images/main_inactive.jpg}
    \includegraphics[]{images/main_active.jpg}
\end{center}
Once you complete the steps in the previous section you should see the main control panel of the GUI pictured above (left). If you are running QMP on port 55555 as in the example, you can just press the connect button without typing anything, and the GUI will automatically connect you to \code{localhost} on port 55555 as seen in the image on the right. Otherwise, you should type in the QMP host and port and click the connect button. From top to bottom, you can see the pause / play button, the simulation time, the simulation state, the QEMU version, and the connect dialog. Next, we can look at some of the more advanced features that the GUI offers.

\section{Features}
The GUI offers many features that aid with debugging and using QEMU in general. The simplest feature is arguably the pause / play button on the main screen, which uses QMP to send "stop" and "continue" signals to the simulation. The following section will explore some of the more advanced features of the GUI such as the CPU Register View and the Memory Dump View. Keep in mind that some features of the GUI make use of not only QMP commands, but also HMP commands. To invoke HMP commands through QMP, the GUI uses the following syntax \code{\{"execute": "human-monitor-command", "arguments": \{"command-line": [HMP\_COMMAND]\}\}}.
\subsection{Memory Dump}
\begin{figure}[h]
    \centering
    \includegraphics[width=\linewidth]{images/MemDump.PNG}
    \label{fig:memdump}
\end{figure}

The memory dump view allows users to view the contents of memory. This view displays the memory address, a hexadecimal representation of memory, and an ASCII representation of memory. To open the memory dump view, navigate to Tools $\rightarrow$ Memory Dump.

\subsubsection{Reading Memory}
To read a part of memory, specify the base address in the \textbf{Address} field. Additionally, specify the number of bytes to read in the \textbf{Size} field. If \textbf{Size} exceeds 2048, then it will be set to 2048. Finally, click \textbf{Refresh} to fetch \textbf{Size} bytes of memory starting \textbf{Address}.\par 
It is also possible to scroll up or down to load adjacent regions of memory. 

\subsubsection{Displaying Memory}\label{ReadMem}
This view automatically displays memory as bytes represented as hexadecimal and ASCII values. The \textbf{Grouping} dropdown menu can be used to set the grouping of bytes in the hexadecimal display. \textbf{Grouping} can be either 1, 2, 4, or 8, corresponding to grouping bytes, 2 btyes, 4 btyes, and 8 bytes. \par
On the right-hand side of the display, there are 2 possible selections for endianness: little endian and big endian. Selecting these displays the memory with the selected endianness. The default endianness is little endian. \par
For either of these display changes to take effect, the display must be refreshed which can be accomplished either by clicking \textbf{Refresh} or by having \textbf{Auto Refresh} selected.

\subsubsection{Auto Refresh}
The \textbf{Auto Refresh} checkbox in the top right corner controls whether the display auto refreshes or not. With \textbf{Auto Refresh} selected, the display will refresh at a rate of 10Hz. By default, \textbf{Auto Refresh} is checked.\par
\textbf{Note:} The auto refresh only refreshes the currently displayed region of memory. To load a different region the user must either scroll or follow directions as specified in \ref{ReadMem}

\subsubsection{Searching}
The \textbf{Search} button lets users search for memory at a specified address. If the address is within the currently displayed range, then the address, hexadecimal value at that address, and the ASCII value at that address will be highlighted, and the view will scroll to make them visible. If the address is not within the currently displayed range, then 1024 bytes starting at the specified address will be loaded and the same values will be highlighted.

\subsubsection{Saving}
The \textbf{Save} button allows the user to save the currently displayed range of memory to a file. To enable the \textbf{Save} button, the simulation must be paused.\par
\textbf{Note:} \textbf{Save} uses \code{pmemsave} to save memory to the file, so any grouping or endian display changes will not be present in the file.

\subsection{Memory Tree View}
\begin{figure}[h]
    \centering
    \includegraphics[width=.6\textwidth]{images/MemTree.PNG}
    \label{fig:memtree}
\end{figure}
The Memory Tree View displays the different regions of memory and their address ranges in the form of a tree. To open the memory tree view navigate to Tools $\rightarrow$ Memory Tree. Each region can be expanded to show its subregions. Additionally, double clicking a region opens up a memory dump view with that region displaying.
\newpage
\subsection{Assembly View}
\begin{figure}[h]
    \centering
    \includegraphics[width=.6\textwidth]{images/AssemblyView.PNG}
    \label{fig:asmview}
\end{figure}

The Assembly View shows the current assembly instruction. To open the Assembly View navigate to Tools $\rightarrow$ Assembly View. \par
\textbf{Note:} To properly use this view, the architecture's instruction pointer register must be set in \code{package/constants.py}. By default, \code{\$eip} is used.\par
\textbf{Areas for Improvement:}
\begin{itemize}
    \item A previous version of the GUI explored using QEMU's existing gdbstub functionality to perform be able to step through assembly, rather than just using an HMP command (as is done in the current version of this software). The current version of this software does not include this functionality because it did not work reliably, however, if this functionality is desired (despite its flaws), an experimental version is located under the branch \code{gdb-fixes-in-progress}. In short, the bug preventing this functionality from existing in the main version of this GUI is because the \code{si} and \code{ni} do not always give a stop signal when they finish stepping. Thus, sometimes the display doesn't properly update. This functionality is currently commented out in the main version.
\end{itemize}

\subsection{CPU Register View}
The CPU Register View is used to view simulation CPU registers and their contents. The view can be opened by navigating to (Tools $\rightarrow$ CPU Register View). 

\begin{center}
    \includegraphics[width=90mm]{images/registers_fancy.jpg}
\end{center}

This view offers options to toggle auto-refresh and to save the registers to a separate file. The text mode can be toggled through the Options menu, and offers a less visually pleasing (but more organized and compact) view of the registers. \newline

The CPU Register View is populated by custom QMP command \code{itc-cpureg}. Currently, this command only works for i386. The process for extending this command to other architectures is described later in the \hyperref[sec:itccpureg]{'Added QMP Commands'} section under '\code{itc-cpureg}'.

\subsection{Logging View}
The Logging View allows you to view, save, and filter trace event logs. To open the log navigate to Tools $\rightarrow$ Logging. Whenever you select a logging option, output is stored to a logfile located at the default logging location (\code{qemu.log} in the user's home directory) or at the path specified in file prompt. The view can be seen below.
\begin{center}
    \includegraphics[width=100mm]{images/logging-view.png}
\end{center}
By default, the display will auto-refresh. There are two ways to select events to log: (1) by selecting a specific trace event (in the left dropdown view) or (2) by selecting a log mask (under the trace event dropdown view). The text entry box with the \textbf{Search} button can be used to search through the trace events dropdown view. The \textbf{Start} and \textbf{Stop} buttons can be used to start and stop logging. When logging is started, the log view is also displayed as shown below.
\begin{center}
    \includegraphics[width=100mm]{images/logging-view-populated.png}
\end{center}
\subsection{Timing View}
\begin{figure}[h]
    \centering
    \includegraphics[width=.6\textwidth]{images/TimeMult.PNG}
    \label{fig:timemult}
\end{figure}
The time multiplier graphs real time vs. $\frac{\Delta t_{sim}}{\Delta t_{real}}$ where $t_{sim}$ is the simulation's virtual time, and $t_{real}$ is the real time. To open the time multiplier graph navigate to Tools $\rightarrow$ Time Multiplier.\par
At the top of the display, there are three values to alter the graph: \textbf{Sampling Rate}, \textbf{Window}, and \textbf{Limit}. \textbf{Sampling Rate}, as its name suggests, is the rate at which the simulation's time is sampled. \textbf{Window} is the number of samples used for the moving average. So, a smaller value for \textbf{Window} makes the graph change quicker, while a larger value makes the graph change slower. \textbf{Limit} is the number of seconds to display. After changing any of these values, click \textbf{Refresh} to make the changes take effect. The default value for all three of these values is 10.\par
On the right-hand side there are various data about the graph, including the absolute minimum, absolute maximum, and the minimum, maximum, mean, and median for values currently displayed.
\subsection{Miscellaneous}
\subsubsection{Dark Mode}
Dark mode can be enabled through the preferences dialog which contains the checkbox used to toggle dark mode (Edit $\rightarrow$ Preferences).
\begin{center}
    \includegraphics[]{images/main_active.jpg}
    \includegraphics[]{images/main_active_dark.jpg}
\end{center}
\subsubsection{Plugin Support}
This system uses \href{https://yapsy.readthedocs.io/en/latest/}{Yapsy} as a plugin manager. To add a plugin after implementing it, specify the plugin path in \code{addPlugins} in \code{package/mainwindow.py}. Then also specify how the plugin should be handled.\par
All plugins should be accessible through the Plugins dropdown menu, which can be found by clicking Tools $\rightarrow$ Plugins. Right now, the only plugin is the QMP Display which shows responses to QMP commands, as shown below.

\begin{center}
    \includegraphics[width=80mm]{images/qmp-monitor-view.png}
\end{center}

\section{Added QMP Commands}

\subsection{\code{get-pmem}} \label{GetPmem}
The \code{get-pmem} command requests the contents of a part of physical memory.
\paragraph{Arguments}
\begin{itemize}
    \item \code{addr: int64}
        \subitem The starting address of the requested part of memory.
    \item \code{size: int64} 
        \subitem The number of bytes of memory being requested.
    \item \code{hash: int64} 
        \subitem A unique identifier.
\end{itemize}   
        
\paragraph{Return}
    \code{get-pmem} returns a \code{MemReturn} object
    \subparagraph{\code{MemReturn}}
    \begin{itemize}
        \item \code{hash: int64}
            \subitem The unique identifier passed to \code{get-pmem}.
        \item \code{vals: MemVal[]}
            \subitem An array of \code{MemVal} objects
    \end{itemize}
    
    \subparagraph{\code{MemVal}}
    \begin{itemize}
        \item \code{val: uint64}
            \subitem Single value of memory. The number of bytes \code{val} represents depends on the value of \code{grouping}.
        \item \code{ismapped: bool}
            \subitem Indicates whether that value is from a mapped region of memory or not.
    \end{itemize}
\textbf{Example:} \\
\code{-> \{"execute:" "get-pmem",} \\
\code{        "arguments": \{}\\
\code{            "addr": 4275044352,}\\
\code{"size": 1024,}\\
\code{      "hash": 0,}\\
\code{"grouping": 1 \} \}}   \\
\code{<- \{"return:" \{"hash": 0, "vals": \{"val": 0, "ismapped": true\}, \{"val": 0, "ismapped": true\}, ... \} \}}

\subsection{\code{mtree}}
The \code{mtree} command returns a tree representation of memory regions. 

\paragraph{Arguments} \code{mtree} takes no arguments.

\paragraph{Return} \code{mtree} returns an array of \code{MemoryMapEntry} objects. This array is a depth-first traversal of the tree.
\subparagraph{\code{MemoryMapEntry}}
\begin{itemize}
    \item \code{name: str}
        \subitem The name of the region of memory.
    \item \code{start: int}
        \subitem The start address of the region of memory.
    \item \code{end: int}
        \subitem The end address of the region of memory.
    \item \code{parent: str}
        \subitem The name of the parent region. A value of \code{""} indicates it is a root node. There can be multiple roots.
\end{itemize}

\textbf{Example:}\\
\code{-> \{"execute": "mtree"\}} \\
\code{<- \{"return": \{ \{"name": "memory", "start": 0, "end": -1, "parent": ""\}, \{"name": "system", "start": 0, "end": -1, "parent": "memory"\}, ... \} \}}

\subsection{\code{itc-sim-time}}
The \code{itc-sim-time} command returns the time in nanoseconds given a specified \code{ClockType}. \code{ClockType} is an enum with possible values \code{realtime, virtual, host, and virtual-rt} which all correspond to QEMU's clock type enum in \code{include/qemu/timer.h}.

\paragraph{Arguments}
\begin{itemize}
    \item \code{clock: ClockType}
        \subitem The clock type to use to get the time.
\end{itemize}

\paragraph{Return} \code{itc-sim-time} returns a \code{SimTime} object.
\subparagraph{\code{SimTime}}
\begin{itemize}
    \item \code{time\_ns: int64}
    \subitem The time in nanoseconds.
\end{itemize}

\textbf{Example:}\\
\code{->\{"execute": "itc-sim-time",}\\
\code{"arguments": \{"clock": "virtual"\} \}}\\
\code{<-\{"return": \{"time\_ns": 1040040609 \} \}}

\subsection{\code{itc-time-metric}}
The \code{itc-time-metric} command gives an array containing the current host time and the current virtual time.

\paragraph{Arguments} \code{itc-time-metric} does not take any arguments.

\paragraph{Return} \code{itc-time-metric} returns an array containing 2 \code{SimTime} objects. The first value is the virtual time, and the second value is the host time.

\textbf{Example:}\\
\code{->\{"execute": "itc-time-metric" \}}\\
\code{<-\{"return": \{ \{"time\_ns": 1040040609\}, \{"time\_ns": 1594065163483704\} \} \}}


\subsection{\code{itc-cpureg}}
\label{sec:itccpureg}
The \code{itc-cpureg} command returns all CPU register names and values along with the total CPU register count. In v.0.1 of the GUI, the CPU Register View used a the HMP command \code{info registers} but this necessitated complex parsing in the GUI (since HMP commands are returned a strings, rather than as JSON objects  as in QMP commands) which was very fragile and broke with QEMU updates. It was suggested that a custom QMP command might eliminate this issue, so the \code{itc-cpureg} command was developed. The \code{info registers} command uses a function which dumped the CPU state to a file. This function was copied then modified to create a function to return the CPU state as a JSON object rather than writing it to a file. 

It is important to understand that the \code{itc-cpureg} command is currently only implemented for i386 systems; since CPU registers vary across architectures, there is no one-size-fits-all solution for all architectures. To extend this command to work for other architectures, a function  \code{[ARCH]\_cpu\_return\_state} needs to be written and connected to the CPUClass function \code{return\_state}. This function will look very similar to that architecture's \code{[ARCH]\_cpu\_dump\_state} function, except it will return the CPU state instead of writing it to a file. Currently, \code{x86\_cpu\_return\_state} is the only example to follow in writing this function. Once this function is written and connected to \code{CPUClass}'s \code{return\_state} function, the \code{itc-cpureg} command will work for that architecture.


\paragraph{Arguments} \code{itc-cpureg} does not take any arguments.

\paragraph{Return} \code{itc-cpureg} returns an array containing a \code{CpuReturn} object and an integer representing the total number of registers described in the array.
 
 
\section{Known Bugs and Possible Improvements}

In inspecting an upgrading this software from v.0.1 to v.0.2, several bug fixes and improvements were identified and made, however, some work remains to further improve this application. This includes:

\begin{itemize}
    \item The version of the Assembly View which uses GDB (described in the section on the Assembly View) could be debugged to enable stepping.
    \item In the QMP Wrapper (\code{qmpwrapper.py}), socket communications with QEMU could be made more robust by replacing the check for termination characters at the end of received messages with a full-featured messaging framework to ensure complete and reliable reception of packets from QEMU.
    \item A known race condition around accessing the window's scroll bar exists in the Memory Dump View. Its impact has been reduced by tightening the window of access to this variable, therefore tightening the critical section but this race condition could still theoretically occur. A more thorough investigation of the state of this race condition and potentially a refactor of \code{memdump.py} would solve this.
\end{itemize} 

\end{document}
